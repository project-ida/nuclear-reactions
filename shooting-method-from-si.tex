
\subsubsection*{Modeling unstable states}


In this section we consider issues associated with the calculation of unstable two-body states.  As above we are interested
in solutions to the radial Schr\"odinger equation

\begin{equation}
E P
~=~
\bigg (
- {\hbar^2 \over 2 \mu} {d^2 \over dr^2}
+ {\hbar^2 \over 2 \mu r^2}
+ V_C
+ V_N
\bigg )
P
\end{equation}

\noindent
subject to boundary conditions

\begin{equation}
%\begin{split}
P(0) ~=~ 0
$$
$$
%\\
P(r) ~\to~ A e^{i k r}
%\end{split}
\end{equation}


We need to develop a normalized version of the problem.  Then since we have a boundary condition at large $r$, we want to integrate inwards to match the boundary condition at the origin. Since there are both real and imaginary parts, we require two degrees of
freedom to arrange for matching both boundary conditions.

It is simplest to work in units of fm and MeV. We can write

\begin{equation}
{\hbar^2 \over 2 \mu ~{\rm fm}^2}
~=~
I_H {m_e \over \mu} {a_0^2 \over {\rm fm}^2}
\end{equation}

\noindent
Evaluating for the cases of interest

\begin{table}[h!]

\begin{center}
{\begin{tabular}{ccccccc}
channel & constant (MeV) \cr
\hline
d+d      & 20.7601 \cr
3+1      & 27.6661 \cr
p+t      & 27.6809 \cr
n+$^3$He  & 27.6512 \cr
\end{tabular}}
\end{center}

\end{table}

\noindent
Then the Hamiltonian is

\begin{equation}
E P
~=~
\bigg \lbrace
{\hbar^2 \over 2 \mu~{\rm fm}^2} \bigg ( -{d^2 \over d \rho^2} + {1 \over \rho^2} \bigg )
+ V_C
+ V_N
\bigg \rbrace
P
\end{equation}

\noindent
or

\begin{equation}
{E \over {\rm MeV}} P
~=~
\bigg \lbrace
{\hbar^2 \over 2 \mu~{\rm fm}^2} \bigg ( -{d^2 \over d \rho^2} + {1 \over \rho^2} \bigg )
+ {V_C \over {\rm MeV}}
+ {V_N \over {\rm MeV}}
\bigg \rbrace
P
\end{equation}

\noindent
where

\begin{equation}
\rho ~=~ {r \over {\rm fm}}
\end{equation}

\noindent
We can take another step and define

\begin{equation}
\epsilon ~=~ {E \over {\rm MeV}}
\ \ \ \ \ \ \
v_N ~=~ {V_N \over {\rm MeV}}
\ \ \ \ \ \ \
v_C ~=~ {V_C \over {\rm MeV}}
\end{equation}

\noindent
Then we get

\begin{equation}
\epsilon P
~=~
\bigg \lbrace
{\hbar^2 \over 2 \mu~{\rm fm}^2 ~{\rm MeV}} 
\bigg ( -{d^2 \over d \rho^2} + {1 \over \rho^2} \bigg )
+ v_C 
+ v_N 
\bigg \rbrace
P
\end{equation}

\noindent

For the bare Coulomb interaction we have

\begin{equation}
%\begin{split}
v_C
~=~ 
{1 \over {\rm MeV}~{\rm fm}} {e^2 \over 4 \pi \epsilon_0 \rho }
$$
$$
%\\
~=~
{2 I_H \over {\rm MeV}} {a_0 \over {\rm fm}} {1 \over \rho}
$$
$$
%\\
~=~
1.43996~ {1 \over \rho}
%\end{split}
\end{equation}

For the discretization the thought is to use simple 3-point differencing.  We start with

\begin{equation}
\epsilon P
~=~
\bigg \lbrace
A \bigg ( -{d^2 \over d \rho^2} + {1 \over \rho^2} \bigg )
+ 
{B \over \rho} 
+ v_N 
\bigg \rbrace
P
\end{equation}

\noindent
We discretize and write

\begin{equation}
\epsilon P_j
~=~
- A {P_{j+1} - 2 P_j + P_{j-1} \over h^2}
+
{A \over \rho_j^2} P_j
+ 
{B \over \rho_j} P_j 
+ 
v_N P_j
\end{equation}

To develop a boundary condition at large $r$ we start with

\begin{equation}
E ~=~ {\hbar^2 k^2 \over 2 \mu}
\end{equation}

\noindent
or

\begin{equation}
k ~=~ \sqrt{2 \mu E \over \hbar^2}
\end{equation}

\noindent
For the normalized version of the problem we have

\begin{equation}
\epsilon ~=~ A \kappa^2
\end{equation}

\noindent
or

\begin{equation}
\kappa ~=~ \sqrt{\epsilon \over A}
\end{equation}

\noindent
and

\begin{equation}
P_{N  } ~=~ e^{i \kappa \rho_{N  }}
\ \ \ \ \ \ \
P_{N+1} ~=~ e^{i \kappa \rho_{N+1}}
\end{equation}

For inward integration we start with

\begin{equation}
A {P_{j-1} \over h^2}
~=~
- A {P_{j+1} - 2 P_j  \over h^2}
+
{A \over \rho_j^2} P_j
+ 
{B \over \rho_j} P_j 
+ 
v_N P_j
-
\epsilon P_j
\end{equation}

\noindent
and rewrite it as

\begin{equation}
P_{j-1} 
~=~
2 P_j - P_{j+1}
+
{h^2 \over A}
\bigg (
{A \over \rho_j^2} 
+ 
{B \over \rho_j}  
+ 
v_N
-
\epsilon 
\bigg )
P_j
\end{equation}


To find the optimum, we select different values for Re$\lbrace \epsilon \rbrace$ and
Im$\lbrace \epsilon \rbrace$.  One approach is to select increments and decrements in both, and pick whichever minimizes

\begin{equation}
|P_0| ~=~ \sqrt{ ({\rm Re} \lbrace P_0 \rbrace)^2 + ({\rm Im} \lbrace P_0 \rbrace)^2}
\end{equation}

\noindent
This is equivalent to a 2-D search based on minimizing a scalar quantity.  We could start with

\begin{equation}
\epsilon ~=~ 0
\end{equation}
  
\noindent
as an initial condition on the energy.
